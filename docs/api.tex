\documentclass[12pt]{article}

\title{FarmClash - API Blueprint for Web-based Idle Game}
\author{Camille De Vuyst, Joren Van der Sande, Thomas De Volder,\\ Faisal Ettarrahi, Ferhat Van Herck, Siebe Mees}
\date{\today}

\begin{document}

\maketitle
\tableofcontents
\newpage

\section{Introduction}
This document outlines the API endpoints for a web-based idle game. Each endpoint requires user authentication and certain routes are restricted to admin users only.

\subsection{Base URL}
The base URL for all API requests is \texttt{https://team3.ua-ppdb.me/api/}.

\section{Users}
\begin{itemize}
    \item Endpoint: \texttt{/users}
    \item Method: GET
    \item Description: Get a look at all the users.
    \item Access: Requires \texttt{admin} user privileges.
    \item Response: A JSON array of user objects, each containing:
    \begin{verbatim}
[
  {
    "created_at": "Date",
    "email": "email@example.com",
    "password": "pbkdf2:sha256:hasedpassword",
    "username": "username"
  },
  {
    "created_at": "Another date",
    "email": "anotheremail@example.com",
    "password": "pbkdf2:sha256:anotherhashedpassword",
    "username": "anotherusername"
  }
  // Additional user objects as needed
]
    \end{verbatim}
\end{itemize}


\section{Maps}
\begin{itemize}
    \item Endpoint: \texttt{/maps}
    \item Method: GET
    \item Description: Get a look at all the maps.
    \item Access: Requires \texttt{admin} user privileges.
    \item Response: A JSON array of map objects, each containing:
    \begin{verbatim}
[
  {
    "map_id": 1,
    "username_owner": "ownerUsername",
    "width": 100,
    "height": 100,
    "created_at": "Tue, 26 Mar 2024 13:55:30 GMT"
  },
  {
    "map_id": 2,
    "username_owner": "anotherOwnerUsername",
    "width": 150,
    "height": 150,
    "created_at": "Wed, 27 Mar 2024 14:00:00 GMT"
  }
  // Additional map objects as needed
]
    \end{verbatim}
\end{itemize}

\section{Resources}
\subsection{General Resource Information}
\begin{itemize}
    \item Endpoint: \texttt{/resources}
    \item Method: GET
    \item Description: Retrieves a comprehensive list of resources across all users. Designed for administrative use to monitor and manage resources system-wide.
    \item Access: Admin privileges required.
    \item Response: A JSON array of objects, each representing a resource. Each object includes the resource type, amount, owner's username, and a unique resource identifier. Example of a generalized response:
    \begin{verbatim}
[
  {
    "resource_id": "unique_identifier",
    "resource_type": "type_of_resource",
    "amount": "quantity_of_resource",
    "owner": "username_of_owner"
  },
  ... // Additional resource objects
]
    \end{verbatim}
\end{itemize}
\subsection{Specific Resource Information}
\begin{itemize}
    \item Endpoint: \texttt{/resources/<string:username>}
    \item Method: GET
    \item Description: Retrieves a list of resources owned by a specific user.
    \item Access: Admin privileges required.
    \item Response: A JSON array of objects, each representing a resource owned by the specified user. Each object includes the resource type, amount, and a unique resource identifier. Example of a generalized response:
    \begin{verbatim}
[
  {
    "resource_id": "unique_identifier",
    "resource_type": "type_of_resource",
    "amount": "quantity_of_resource"
  },
  ... // Additional resource objects
]
    \end{verbatim}
\end{itemize}

\section{Terrain Map}

\subsection{General Terrain Map Information}
\begin{itemize}
    \item Endpoint: \texttt{/terrain-map}
    \item Method: GET
    \item Description: Get a look at all the terrain tiles in the map.
    \item Access: Requires \texttt{logged in} user privileges.
    \item Response: A JSON array of tile objects, each containing:
    \begin{verbatim}
    [
      {
        "map_height": 2,
        "map_width": 2,
        "terrain_tiles": [
          [Water1.1, Water1.2],
          [Water2.1, Water2.2]
        ]
      }
    ]
    \end{verbatim}
\end{itemize}

\subsection{Specific Terrain Tile Information}
\begin{itemize}
    \item Endpoint: \texttt{/terrain-map/<string:friend\_username>}
    \item Method: GET
    \item Description: Get a look at a user specific terrain tile in the map.
    \item Access: Requires \texttt{logged in} user privileges.
    \item Response: A JSON object containing the terrain tile information:
    \begin{verbatim}
    [
      {
        "map_height": 2,
        "map_width": 2,
        "terrain_tiles": [
          [Water1.1, Water1.2],
          [Water2.1, Water2.2]
        ]
      }
    ]
    \end{verbatim}
\end{itemize}

\section{Friendships}
\begin{itemize}
    \item Endpoint: \texttt{/friends}
    \item Method: GET
    \item Description: Get a list of all friends for the current user.
    \item Access: Requires \texttt{logged in} user privileges.
    \item Response: A JSON array of relationship objects, each containing:
    \begin{verbatim}
[
  "friend1",
  "friend2",
  "friend3"
]
    \end{verbatim}
\end{itemize}

\section{Chat}
\begin{itemize}
    \item Endpoint: \texttt{/messages/<string:friend\_name>}
    \item Method: GET
    \item Description: Get a list of all chat messages for the current user.
    \item Access: Requires \texttt{logged in} user privileges.
    \item Response: A JSON array of chat message objects, each containing:
    \begin{verbatim}
[
  {
    message_id: 1,
    "sender": "senderUsername",
    "receiver": "receiverUsername",
    "message": "Hello, how are you?",
    "created_at": "Tue, 26 Mar 2024 13:55:30 GMT"
  },
  {
    message_id: 2,
    "sender": "receiverUsername",
    "receiver": "senderUsername",
    "message": "I'm good, thanks!",
    "created_at": "Tue, 26 Mar 2024 13:56:30 GMT"
  }
  // Additional chat message objects as needed
]
    \end{verbatim}
\end{itemize}

\section{Leaderboard}
\begin{itemize}
    \item Endpoint: \texttt{/leaderboard}
    \item Method: GET
    \item Description: Get a list of top 3 users, 2 friends and yourself sorted by their score.
    \item Access: Requires \texttt{logged in} user privileges.
    \item Response: A JSON array of user objects, each containing:
    \begin{verbatim}
[
  {
    "place": 1,
    "username": "username",
    "score": 100
  },
  {
    "place": 2,
    "username": "anotherUsername",
    "score": 200
  }
  // Additional user objects as needed
]
    \end{verbatim}
\end{itemize}


\section{Market}
\begin{itemize}
    \item endpoint: {@game\_blueprint.route('/update-building-map', methods=['POST'])}
        \item{Method:} POST
        \item{Description:} Handles POST requests to insert JSON data into the database. Expects JSON data in the request body.
        \item {Access:} Requires \texttt{logged in} user privileges.
        \item {Response:} A JSON object with status and message indicating success or failure.
    \end{itemize}

\section{Building}
\begin{itemize}
    \item endpoint {@game\_blueprint.route('/fetch-building-information', methods=['GET'])}
        \item {Method:} GET
        \item {Description:} Handles GET requests to fetch building information for the current user.
        \item {Access:} Requires \texttt{logged in} user privileges.
        \item {Response:} A JSON object containing building information.
    \end{itemize}


\section{Animals}
\subsection{General Animal Information}
\begin{itemize}
    \item Endpoint: \texttt{/animals}
    \item Method: GET
    \item Description: Retrieves a comprehensive list of animals for a user.
    \item Access: Requires logged in user privileges
    \item Response: A JSON array of objects, each representing an animal. Each object includes the animal's specie, amount, owner's username, and a when it was last updated by the an 'idle' update. Example of a generalized response:
    \begin{verbatim}
[
  {
    "species": "specie_of_animal",
    "owner": "username_of_owner",
    "amount": "quantity_of_animal",
    "last_updated": "isoformat_of_datetime"
  },
  ... // Additional animal objects
]
    \end{verbatim}
\end{itemize}

\subsection{General Animal Update Information}
\begin{itemize}
    \item Endpoint: \texttt{/update-animals}
    \item Method: POST
    \item Description:updated list of animals for a user. This endpoint is used to update the quantity of animals owned by a user.
    \item Access: Requires logged-in user privileges.
    \item Request: A JSON object with general information about the update.
    The object includes an update type which describes why the amount was updated (Useful for Idle updates) and a species JSON object which includes an array for each animal. The first element represents wheter or not the amount was updated from the one stored in the database and the second element containing the new value if it was updated (It is done like this remove redundent update DB calls and to not have to call fetches for each animal apart. Here are a few examples.
    \begin{verbatim}
{
    "update_type": 'idle'
    "species":
    {
	       "Pig": [False]
	       "Cow": [True, 14]
	       "Chicken": [True, 12]
	       "Goat": [True,17]
    }
}
 ...// 'idle' is used when last_updated needs to be reset aswell this is to make the idle animal increments from their appropriate buildings easier
{
    "update_type": 'explore'
    "species":
    {
	       "Pig": [True, 11]
	       "Cow": [False]
	       "Chicken": [False]
	       "Goat": [True,17]
    }

}

{
    "update_type": 'attack'
    "species":
    {
	       "Pig": [True, 5]
	       "Cow": [False]
	       "Chicken": [False]
	       "Goat": [True,8]
    }

}
    \end{verbatim}
    \item Response: A JSON object indicating the status of the update operation.
    \begin{verbatim}
{
  "status": "success/error",
  "message": "Description of the outcome of the operation"
}
    \end{verbatim}
\end{itemize}

\end{document}
