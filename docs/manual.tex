\documentclass[12pt]{article}

% Package for including images
\usepackage{graphicx}
% Package for database diagrams and other diagrams
\usepackage{tikz}
% Package for hyperlinks
\usepackage{hyperref}

\title{Software Product Technical Manual}
\author{Your Name and Team Members' Names}
\date{\today}

\begin{document}

\maketitle
\tableofcontents
\newpage

\section{Introduction}
\subsection{Purpose}
This document serves as the comprehensive manual for the Software Product, detailing the design, database schema, and functionality to assist developers, testers, and users in understanding the system.

\subsection{Scope}
Include information about the scope of this manual, detailing the software product's intended functionalities and the audience for this document.

\section{System Overview}
Describe the overall design of the program, including a high-level architecture diagram and a description of the main components.

\section{Database Design}
\subsection{Database Schema}
Provide a detailed database diagram showing the tables, relationships, and keys. Describe each element briefly.
% Example for including an image (Your diagram)
% \begin{figure}[h]
%   \centering
%   \includegraphics[width=0.8\textwidth]{path/to/your/database/diagram.png}
%   \caption{Database schema diagram}
% \end{figure}

\subsection{Table Descriptions}
For each table in the database, provide a detailed description, including the purpose of the table, a description of each field, and the relationships to other tables.

\section{Functionality Description}
\subsection{Login Feature}
Detail the implementation of the Login feature, including any relevant code snippets, algorithms, or flowcharts.
% Example for including a code snippet
% \begin{verbatim}
%   Code snippet here
% \end{verbatim}

\subsection{Other Features}
Describe other features of the software, following the same format as the Login feature section. Include diagrams or flowcharts if applicable.

\subsubsection{Map Navigation}
In this feature, the user is able to navigate the map using both keyboard and mouse controls. Specifically, the user can move using the arrow keys to navigate the map by one square at a time. This functionality can be tested by simply pressing one arrow key at a time. Additionally, the user can also move the map by left-clicking on a non-interactive part of the map (e.g., where no buildings or buttons are placed) and dragging the map in the desired direction.

\section{Development and Implementation}
Discuss how the project was developed and implemented, emphasizing the involvement of each team member in various phases such as coding, testing, and documentation.

\section{Conclusion}
Summarize the key points of the manual and the software product's capabilities.

\end{document}
