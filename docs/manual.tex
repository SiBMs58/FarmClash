\documentclass[12pt]{article}

% Package for including images
\usepackage{graphicx}
% Package for database diagrams and other diagrams
\usepackage{tikz}
% Package for hyperlinks
\usepackage{hyperref}

\title{Software Product Technical Manual}
\author{Your Name and Team Members' Names}
\date{\today}

\begin{document}

\maketitle
\tableofcontents
\newpage

\section{Introduction}
\subsection{Purpose}
This document serves as the comprehensive manual for the Software Product, detailing the design, database schema, and functionality to assist developers, testers, and users in understanding the system.

\subsection{Scope}
Include information about the scope of this manual, detailing the software product's intended functionalities and the audience for this document.

\section{System Overview}
Describe the overall design of the program, including a high-level architecture diagram and a description of the main components.

\section{Database Design}
\subsection{Database Schema}
% Add your database diagram here
% \begin{figure}[h]
%   \centering
%   \includegraphics[width=0.8\textwidth]{path/to/your/database/diagram.png}
%   \caption{Database schema diagram}
% \end{figure}

\subsection{Table Descriptions}
\subsubsection{Buildings Table}
The Buildings table stores information about buildings in the game.
\begin{itemize}
    \item \textbf{building\_id}: Primary key identifying the building.
    \item \textbf{username\_owner}: Username of the owner of the building.
    \item \textbf{farm\_id}: Foreign key referencing the farm where the building is located.
    \item \textbf{building\_type}: Type of the building.
    \item \textbf{level}: Level of the building (default is 1).
    \item \textbf{x, y}: Coordinates of the building on the map.
    \item \textbf{tile\_rel\_locations}: JSONB field storing relative tile locations of the building.
    \item \textbf{created\_at}: Timestamp indicating when the building was created.
\end{itemize}

\subsubsection{Market Table}
The Market table stores dynamic pricing information for crops in the game.
\begin{itemize}
    \item \textbf{crop\_name}: Primary key representing the name of the crop.
    \item \textbf{current\_price}: Current price of the crop.
    \item \textbf{current\_quantity\_crop}: Current quantity of the crop in the market.
    \item \textbf{prev\_quantity\_crop}: Previous quantity of the crop in the market.
    \item \textbf{last\_update}: Timestamp indicating when the market data was last updated.
\end{itemize}

\section{Functionality Description}
\subsection{Update Market Functionality}
This functionality handles POST requests to insert market count data into the database. The server updates the market data for the specified crop based on the received JSON data.

\subsection{Fetch Crop Price Functionality}
This functionality handles GET requests to fetch the price of a crop from the market. The server queries the database to retrieve the price of the specified crop from the market.

\section{Development and Implementation}
Discuss how the project was developed and implemented, emphasizing the involvement of each team member in various phases such as coding, testing, and documentation.

\section{Conclusion}
Summarize the key points of the manual and the software product's capabilities.

\end{document}
