\documentclass[12pt]{article}
\usepackage[utf8]{inputenc}
\usepackage{geometry}
\usepackage{graphicx}

\geometry{a4paper}

\title{Technical Report - FarmClash}
\author{Camille De Vuyst, Joren Van der Sande, Thomas De Volder,\\ Faisal Ettarrahi, Ferhat Van Herck, Siebe Mees}
\date{\today}

\begin{document}

\maketitle
\tableofcontents
\newpage

\begin{abstract}
This technical report describes the development and implementation of FarmClash, a web-based idle game designed for the Programming Project Databases course at the University of Antwerp. The game integrates concepts of resource management and multiplayer interaction within a farm management context, drawing inspiration from popular titles like Clash of Clans and Hay Day.
\end{abstract}

\section{Introduction}
FarmClash is an educational project designed to apply database and web development skills in a practical scenario. Inspired by games like Clash of Clans and Hay Day, it challenges players to manage a farm, interact with other players, and strategize in real-time, balancing competitive and cooperative gameplay elements.

\section{Project Objectives}
The primary objectives of this project include:
\begin{itemize}
    \item Develop a functional web-based idle game, inspired by titles such as Grepolis and Clash of Clans, focusing on the management of cities and resources.
    \item Utilize PostgreSQL to manage a robust database that handles game states, user data, and interactive features such as guilds and direct messaging.
    \item Implement secure and efficient server-side interactions using Flask, ensuring that the system handles real-time data exchange and multiplayer interactions seamlessly.
    \item Create an intuitive user interface that facilitates easy navigation and management of game tasks, utilizing technologies such as React for dynamic frontend development and CSS frameworks for aesthetic design.
    \item Design the game to support background updates and cooldowns effectively, representing idle game mechanics where actions continue to progress even when players are not actively engaged.
    \item TODO: Ensure that the application adheres to modern web standards and security measures, including the use of JSON Web Tokens for stateless authentication and secure communication.
    \item Emphasize software extensibility and maintainability, preparing the system for future upgrades and scalability.
    \item Deploy and maintain the software system on a production environment, leveraging provided Google Cloud Platform resources for hosting.
\end{itemize}
This project is designed to meet the educational goals of learning to develop extensive software projects within a team setting, discovering and integrating new technologies, and effectively planning and distributing tasks.


\section{Game Description}
FarmClash allows players to build and manage their farms, defend against attacks from other players, and maximize their profits through strategic selling of farm produce influenced by dynamic market prices. Players start with a basic farm and can upgrade their facilities and defenses as they progress, interacting with other players through alliances and competitive gameplay.

\section{Technologies Used}
\begin{itemize}
    \item \textbf{Flask:} Serves as the web server framework.
    \item \textbf{PostgreSQL:} Manages all data storage needs.
    \item \textbf{Redis:} Implemented for caching purposes to enhance performance (pending verification).
    \item \textbf{JavaScript:} Used for dynamic frontend development.
    \item \textbf{Tools:} Includes PyCharm, DataGrip, JIRA, and Google Cloud among others.
\end{itemize}

\section{Implementation Details}
\subsection{Database Schema}
The database schema is designed to efficiently handle game data and user interactions. It includes tables for users, game states, transactions, and market prices, reflecting the dynamic nature of the game's economy.
\\ $\ast$ TODO: Add more details here. e.g. image of the final db schama

\subsection{Backend Logic}
The backend handles requests from the client, processes game logic, and interacts with the database. It ensures that all player actions are reflected in real-time and manages the continuous accumulation of resources even when the player is offline.
\subsubsection{Login and Authentication Feature}
The login and authentication feature is implemented using secure handling of credentials and session management. We utilize JSON Web Tokens (JWT) to maintain user sessions, ensuring that users are authenticated and authorized to perform actions based on their roles. Passwords are hashed using modern cryptographic methods before storage to enhance security.
\subsubsection{Friends and Messaging Feature}
The friends and messaging features are central to fostering a community environment and enabling social interactions within the game.
\paragraph{Friends Feature:}
The backend supports functionalities related to the management of friendships between users. Users can add new friends, and view their list of friends. The implementation involves:
\begin{itemize}
    \item \textbf{Friend Request Management:} Users add friends using a specific API endpoint that updates the database to include a new friendship. This is implemented in the endpoint defined in \texttt{api.py} (see line 96). The endpoint is:
    \begin{verbatim}
    .../add_friend/<string:friend_name>
    \end{verbatim}
    By utilizing the User Data Access, we retrieve data for both the current user and the prospective friend, converting this information into user objects through the get\_user function. Once we have the user data, we create a new friendship by employing the add\_friendship method from the Friendship Data Access class.
    \item \textbf{List of Friends:} Users can retrieve their list of friends through an API call, which fetches the data from the database and returns it in a formatted JSON array. This feature ensures that users can easily access and interact with their friends list. The relevant API call is implemented in \texttt{views/friends.py} (see line 43). The endpoint is:
    \begin{verbatim}
    .../api/friends
    \end{verbatim}
\end{itemize}
\paragraph{Messaging Feature:}
\\ $\ast$ TODO
These features not only enhance user engagement but also contribute to the game's social dynamics, making the gaming experience more interactive and enjoyable for users.

\subsection{Frontend Interface}
The frontend provides an interactive and user-friendly interface using React, allowing players to engage with the game seamlessly. It features a map view for navigating different parts of the farm and a market interface for selling produce.
\subsubsection{Login and Authentication Feature}
\subsubsection{Friends and Messaging Feature}

\section{Challenges and Solutions}
Throughout the development, the team encountered and overcame numerous challenges, such as optimizing database queries and ensuring smooth synchronous player interactions. Specific issues included managing the real-time update of resource levels and implementing secure authentication mechanisms.

\section{Results and Testing}
The game was rigorously tested to ensure functionality across different systems and scenarios. Our testing approach was comprehensive, incorporating automated unit tests, integration tests, and user acceptance testing to cover both the backend and frontend components of our application.
\subsection{Backend Testing}
For backend testing, we utilized Pytest, a powerful and flexible testing tool. Our tests included:
\begin{itemize}
    \item \textbf{Database Connection Tests:} Using \texttt{test\_dbconnection.py}, we verified the stability and reliability of our database connections, ensuring that all interactions with the database handled data correctly and maintained integrity under various conditions.
    \item \textbf{API Functionality Tests:} We conducted extensive testing on our RESTful API endpoints to ensure accurate response statuses and proper data formatting. This included testing:
        \begin{itemize}
            \item \textbf{User Endpoint:} Verified that only administrators could retrieve user data, and that the data was correctly formatted as JSON.
            \item \textbf{Maps Endpoint:} Ensured that map data could only be accessed by administrators, testing for correct map attributes and response format.
            \item \textbf{Resources Endpoint:} Checked both general and specific resource retrieval endpoints for proper authorization checks and JSON formatting.
            \item \textbf{Terrain Map Endpoint:} Tested user-specific access to terrain maps, ensuring the integrity and format of the terrain data returned.
            \item \textbf{Friendships Endpoint:} Assured that users could reliably retrieve their list of friends, with responses properly formatted as JSON arrays.
            \item \textbf{Chat Endpoint:} Confirmed functionality for retrieving chat messages between users, focusing on correct data handling and security measures.
        \end{itemize}
    These tests not only verified proper access control and data integrity but also ensured that our server effectively handled errors and returned appropriate status codes under various scenarios.
    \item \textbf{Data Access Layer Tests:}
    \begin{itemize}
        \item \textbf{User Data Access:} Using \texttt{test\_user\_data\_access.py}, we tested CRUD operations for user data to ensure accurate storage, retrieval, updating, and deletion of user information.
        \item \textbf{Map Data Access:} In \texttt{test\_map\_data\_access.py}, we verified the functionality of our map management system, ensuring that map data manipulations were handled correctly.
        \item \textbf{Tile Data Access:} The \texttt{test\_tile\_data\_access.py} allowed us to ensure that tile-based operations, crucial for the game's map functionality, were accurate and efficient.
        \item \textbf{Resource Data Access:} Through \texttt{test\_resource\_data\_access.py}, we tested the handling of game resources, confirming the correct implementation of resource accumulation and usage.
        \item \textbf{Friendship Data Access:} With \texttt{test\_friendship\_data\_access.py}, we assessed the systems managing player interactions and relationships within the game.
        \item \textbf{Chat Message Data Access:} Using \texttt{test\_chatmessage\_data\_access.py}, we evaluated the functionality of in-game chat systems, ensuring reliable and secure message delivery and storage.
    \end{itemize}
    \item \textbf{Integration Tests:} We conducted extensive tests to ensure that these individual components functioned together seamlessly, simulating real-world usage to detect any integration issues.
\end{itemize}
These targeted tests helped us to systematically validate each aspect of our backend, ensuring robustness and reliability throughout the game's infrastructure.

\subsection{Frontend Testing}
Frontend testing was conducted using Jasmine, focusing on the interactive aspects of our application:
\begin{itemize}
    \item \textbf{UI Component Tests:} Using \texttt{canvas-tests.js}, we tested the rendering and behavior of graphical components, such as the game map and resource widgets, to ensure they behaved consistently across different browsers and resolutions.
    \item \textbf{User Interaction Tests:} We simulated user interactions such as clicking, dragging, and keyboard inputs to ensure the UI responded correctly and efficiently without errors or unexpected behavior.
    \item \textbf{Performance Tests:} To assess the application’s performance, particularly during peak load times, we measured response times and resource usage to identify and mitigate any potential bottlenecks.
\end{itemize}
\subsection{User Acceptance Testing}
User acceptance testing was carried out with a group of target users who provided valuable feedback on the usability and overall experience of the game. This feedback was crucial in refining the gameplay mechanics and interface, leading to several iterations that enhanced user engagement and satisfaction.
\subsection{Continuous Integration}
We integrated continuous integration tools into our development process, allowing us to automatically run tests upon every commit to our version control system. This helped us quickly identify and rectify issues early in the development cycle, improving product quality and reducing time to deployment.
Overall, our structured and thorough approach to testing ensured that 'FarmClash' was robust, user-friendly, and scalable, ready to handle the demands of real-world usage by players across the globe.

\section{User Manual}
\subsection{Installation}
Details the steps required to install and run the game locally, as outlined in the README document.

\subsection{Gameplay}
Explains basic gameplay mechanics, how to interact within the game, and strategies for new players. It covers logging in, starting a farm, upgrading buildings, and interacting with other players.

\section{Conclusion}
The project successfully demonstrates the application of database and web development skills in creating a functional and engaging multiplayer game. Future enhancements could include more complex defense strategies and a broader range of market dynamics.

\section{References}
List all references and resources used during the development of the project.

\end{document}
